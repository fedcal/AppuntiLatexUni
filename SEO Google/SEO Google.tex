\documentclass[a4paper]{extarticle}
\usepackage[utf8]{inputenc}
\usepackage{hyperref}
\usepackage{geometry}
\usepackage{fancyhdr}
\usepackage{graphicx} % libreria per le immagini
\usepackage{amssymb} %libreria per i simboli (ex. alfabeto reco)
\usepackage{algorithm2e} %libreria per scrivere pseudocodice
\usepackage{longtable}
\usepackage{caption}
\usepackage{lastpage}
\usepackage{adjustbox}

\setlength{\parindent}{0em}%indentazione paragrafo
\setlength{\parskip}{1em}%spazio tra paragrafi
\renewcommand{\baselinestretch}{1.3}%interlinea
\graphicspath{ {./} }
\geometry{
    a4paper,
    left=10mm,
    right=10mm,
    bottom=20mm
}


\hypersetup{
    colorlinks=true,
    linkcolor=blue,
    filecolor=blue,      
    urlcolor=blue,
    pdftitle={Overleaf Example},
    pdfpagemode=FullScreen
}

\pagestyle{fancy}
\fancyhf{}
\rhead{Federico Calò}
\lhead{SEO Google}
\cfoot{  \thepage }


\title{SEO Google}
\author{\href{http://www.federicocalo.it}{Federico Calò} }
\date{}

\begin{document}
\maketitle
\newpage
\tableofcontents
\voffset -30pt

\newpage
\section{Introduzione}
Il termine SEO è l’acronimo inglese dell’espressione Search Engine Optimization e si riferisce all’insieme di strategie e di attività che hanno l’obiettivo di ottimizzare un sito web per i motori di ricerca (Google su tutti) in modo tale da garantirgli visibilità e notorietà. Concretamente, il fine della SEO è fare in modo che il sito web (o l’e-commerce) si posizioni nei primi posti delle SERP, ovvero i risultati di ricerca che compaiono sotto forma di elenco dopo aver digitato una query, una domanda. Il consulente SEO è il professionista che decide quali strategie SEO utilizzare per ottimizzare il sito web e quali tecniche adottare di volta in volta per favorirne il posizionamento sui motori di ricerca. Le tecniche SEO da impiegare sono molteplici e, per essere efficaci, devono essere utilizzate in modo sistemico e cooperante e devono essere continuative nel tempo. Interrompere con le attività SEO significa far precipitare il tuo sito web in fondo ai risultati di ricerca, compromettendo tutto il lavoro certosino precedentemente fatto dal SEO Expert, e significa anche che il tuo sito web sarà scavalcato dai concorrenti più agguerriti. Gli obiettivi della seo si possono riassumere in:

\begin{itemize}
\item far visualizzare il tuo sito a nuovi clienti 
\item ti aiuta a risparmiare tempo e denaro
\item migliora nel tempo il posizionamento del sito
\item è uno strumento di visibilità
\item  permette di ottenere un vantaggio sui concorrenti
\end{itemize}

\section{Nozioni di base}
Google è un motore di ricerca completamente automatizzato che utilizza software chiamati web crawler per esplorare regolarmente il Web e trovare siti da aggiungere al suo indice. In effetti, la maggior parte dei siti riportati nei nostri risultati non viene inviata manualmente per l'inclusione, ma viene trovata e aggiunta automaticamente quando i nostri web crawler eseguono la scansione del Web. La Ricerca Google prevede essenzialmente tre fasi:
\begin{itemize}
\item Scansione: Google usa programmi automatizzati chiamati crawler per cercare sul Web pagine nuove o aggiornate. Google memorizza gli indirizzi di tali pagine (o URL delle pagine) in un vasto elenco da esaminare in un secondo momento. Troviamo le pagine con molti metodi diversi, ma quello principale è seguire i link da pagine che già conosciamo.
\item Indicizzazione: Google visita le pagine che ha trovato tramite la scansione e cerca di analizzare il contenuto di ogni pagina. Google analizza i contenuti, le immagini e i file video per cercare di capire l'argomento della pagina. Queste informazioni vengono archiviate nell'Indice Google, un enorme database archiviato su molti computer.
\item Pubblicazione dei risultati di ricerca: quando un utente effettua ricerche su Google, Google cerca di determinare i risultati di qualità ottimale. I risultati "migliori" vengono stabiliti in base a molti fattori, ad esempio la posizione, la lingua, il dispositivo (desktop o telefono) e le query precedenti dell'utente. 
\end{itemize}

Google cerca automaticamente i siti da aggiungere al suo indice; in genere non è necessaria alcuna azione da parte tua, se non pubblicare il sito sul Web. A volte, però, alcuni siti non vengono rilevati. Verifica se il tuo sito è su Google e scopri come rendere più visibili i tuoi contenuti nella Ricerca Google. Per sapere se le pagine del tuo sito sono già state indicizzate, cerca il sito nella Ricerca Google utilizzando una query del seguente tipo: site:nomesito.dominio. Benché Google esegua la scansione di miliardi di pagine, è inevitabile che alcuni siti non vengano rilevati. I motivi più frequenti per cui i nostri crawler non rilevano un sito sono:
\begin{itemize}
\item Non ci sono altri siti sul Web che includono link al tuo sito.
\item Il sito è stato messo online da poco, Google non avrà avuto il tempo di sottoporlo a scansione. Google può impiegare qualche settimana per rilevare un nuovo sito o eventuali modifiche a siti esistenti.
\item La struttura del sito rende difficile a Google eseguire una scansione efficace dei contenuti. Se il sito è basato su altre tecnologie specializzate, anziché su HTML, Google potrebbe avere problemi a eseguirne correttamente la scansione. 
\item Google ha ricevuto un errore durante la scansione del sito. I motivi più comuni sono la presenza di una pagina di accesso per il tuo sito o il fatto che il tuo sito blocchi Google per qualche ragione. 
\item Google non ha rilevato il sito. Benché Google esegua la scansione di miliardi di pagine, è inevitabile che alcuni siti non vengano rilevati, in particolare quelli più piccoli. Aspetta un po' di tempo e prova a fare in modo che altri siti rimandino al tuo tramite link.
\end{itemize}
Se vuoi, puoi aggiungere il tuo sito a Search Console per vedere se c'è un errore che potrebbe impedire a Google di comprendere il tuo sito.

\end{document}
