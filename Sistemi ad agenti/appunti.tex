\documentclass[a4paper]{extarticle}
\usepackage[utf8]{inputenc}
\usepackage{hyperref}
\usepackage{geometry}
\usepackage{fancyhdr}
\usepackage{graphicx} % libreria per le immagini
\usepackage{amssymb} %libreria per i simboli (ex. alfabeto reco)
\usepackage{algorithm2e} %libreria per scrivere pseudocodice
\usepackage{longtable}
\usepackage{caption}
\usepackage{lastpage}
\usepackage{tabto}

\setlength{\parindent}{0em}%indentazione paragrafo
\setlength{\parskip}{1em}%spazio tra paragrafi
\renewcommand{\baselinestretch}{1.3}%interlinea
\graphicspath{ {./} }
\geometry{
    a4paper,
    left=10mm,
    right=10mm,
    bottom=20mm
}
\usepackage{enumitem}

\setlist{leftmargin=5.5mm}


\hypersetup{
    colorlinks=true,
    linkcolor=blue,
    filecolor=blue,      
    urlcolor=blue,
    pdftitle={Overleaf Example},
    pdfpagemode=FullScreen
}

\pagestyle{fancy}
\fancyhf{}
\rhead{Federico Calò}
\lhead{Sistemi ad agenti - Appunti}
\cfoot{  \thepage }


\title{Sistemi ad Agenti - Appunti}
\author{\href{http://www.federicocalo.dev}{Federico Calò} }
\date{}

\begin{document}
\maketitle
\newpage
\tableofcontents
\voffset -30pt

\newpage

\section*{Premessa}

\newpage

\section{Introduzione}

\subsection{Che cos'è l'AI?}

\subsection{I fondamentali dell' Intelligenza Artificiale}

\subsection{La storia dell'Intelligenza Artificiale}

\subsection{Lo stato dell'arte}

\subsection{Rischi e benefici dell'AI}

\newpage


\section{Agenti intelligenti}

\subsection{Gli agenti e gli ambienti}

\subsection{Un buon comportamento: il concetto di razionalità}

\subsection{La natura degli ambienti}

\subsection{La struttura degli agenti}

\newpage

\section{Risoluzione di problemi attraverso la ricerca}

\subsection{La risoluzione dei problemi degli agenti}

\subsection{Esempi di problemi}

\subsection{Algoritmi di ricerca}

\subsection{Strategie di ricerca non informate}

\subsection{Strategie di ricerca informate (euristisca)}

\subsection{Funzioni euristiche}

\newpage

\section{Ricerca in ambienti complessi}

\subsection{Ricerca locale e ottimizzazione dei problemi}

\subsection{Ricerca locale in spazi continui}

\subsection{Ricerca con azioni non deterministiche}

\subsection{Ricerca in ambienti parzialmente osservabili}

\newpage

\section{CSP: problemi di soddisfazione dei vincoli}

\subsection{Definizione dei CSP}

\subsection{Propagazione dei vincoli: Interfacce all'interno dei CSP}

\subsection{Ricerca backtracking per i CSP}

\subsection{Ricerca locale per CSP}

\subsection{La struttura dei problemi}

\newpage

\section{Ricerca e giochi contraddittori}

\subsection{Teoria dei giochi}

\subsection{Decisioni ottimali nei giochi}

\subsection{Albero di ricerca Alpha-Beta euristico}

\subsection{Albero di ricerca Monte Carlo}

\subsection{Giochi stocastici}

\subsection{Giochi parzialmente osservabili}

\subsection{Limitazioni di algoritmi di ricerca dei giochi}

\newpage

\section{Agenti logici}

\subsection{Knowledge Base degli agenti}

\subsection{Il mondo Wumpus}

\subsection{Logica}

\subsection{Proposizioni logiche: Una logica veramente semplice}

\subsection{Dimostrazione di proposizione logiche}

\subsection{Efficace controllo del modello proposizionale}

\subsection{Agenti basati su logica proposizionale}

\newpage

\section{Logica del primo ordine}

\subsection{Rappresentazione rivisitata}

\subsection{Sintassi e semantica della logica del primo ordine}

\subsection{Usi della logica del primo ordine}

\subsection{Ingegneria della conoscenza nella logica del primo ordine}

\newpage

\section{Inferenza nella logica del primo ordine}

\subsection{Inferenza proposizionale vs Inferenza del primo ordine}

\subsection{Unificazione e inferenza del primo ordine}

\subsection{Concatenamento in avanti (Forward Chaining)}

\subsection{Concatenamento all'indietro (Backward Chaining)}

\subsection{Risoluzione}

\newpage

\section{Rappresentazione della conoscenza}

\subsection{Ingegneria ontologica}

\subsection{Categorie e oggetti}

\subsection{Eventi}

\subsection{Oggetti mentali e modelli logici}

\subsection{Sistemi di ragionamento per categorie}


\end{document}




